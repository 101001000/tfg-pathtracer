\documentclass[11pt]{report}
 
\usepackage[x11names,table]{xcolor}
 
\usepackage[spanish]{babel} %Español 
\usepackage{amsmath}
\usepackage{graphicx}
\usepackage{listings}
\usepackage{color}
\usepackage{vmargin}
\usepackage{graphbox}
\usepackage{multirow}

\usepackage{pgf-pie}  
\usepackage{chronology}
\usepackage{float}
\usepackage{pgfplots}
\usepackage[utf8x]{inputenc}
\usepackage[TS1,T1]{fontenc}
\usepackage{array, booktabs}
\usepackage{caption}
\usepackage{subcaption}
\DeclareCaptionFont{blue}{\color{LightSteelBlue3}}

\PassOptionsToPackage{hyphens}{url}
\usepackage[breaklinks=true]{hyperref}

\newcommand{\foo}{\color{LightSteelBlue3}\makebox[0pt]{\textbullet}\hskip-0.5pt\vrule width 1pt\hspace{\labelsep}}
\newcommand{\lstfont}[1]{\color{#1}\scriptsize\ttfamily}

\graphicspath{ {Images/} }
 
%\setmargins{1.5cm}       % margen izquierdo
%{1.5cm}                  % margen superior
%{16cm}                 % anchura del texto
%{23.42cm}                % altura del texto
%{10pt}                   % altura de los encabezados
%{1cm}                    % espacio entre el texto y los encabezados
%{0pt}                    % altura del pie de página
%{2cm}                    % espacio entre el texto y el pie de página


\definecolor{dkgreen}{rgb}{0,0.6,0}
\definecolor{gray}{rgb}{0.5,0.5,0.5}
\definecolor{mauve}{rgb}{0.58,0,0.82}
\definecolor{purple}{rgb}{0.58,0,0.58}

\lstset{frame=single,
	language=C++,
	aboveskip=3mm,
	belowskip=3mm,
	showstringspaces=false,
	columns=flexible,
	basicstyle={\small\ttfamily},
	numbers=none,
	numberstyle=\tiny\color{gray},
	keywordstyle=\color{blue},
	commentstyle=\color{dkgreen},
	stringstyle=\color{mauve},
	breaklines=true,
	breakatwhitespace=true,
	tabsize=4,
	emph={
        cudaMalloc, cudaFree,
        __global__, __shared__, __device__, __host__,
        __syncthreads,
    },
	emphstyle=\color{purple},
	moredelim=[s][\ttfamily]{<<<}{>>>}
}
 
 
\def\code#1{\texttt{#1}}
	 
\setcounter{tocdepth}{5} %shows all levels incl. paragraph
\begin{document}
 


\begin{center}
    \rule[0.5ex]{\linewidth}{2pt}\vspace*{-\baselineskip}\vspace*{3.2pt}\\
    \rule[0.5ex]{\linewidth}{2pt}\\
    [2mm]
    {\textbf{\LARGE{Evaluación y aceleración del algoritmo Path tracing en arquitecturas heterogéneas}} }\\[3mm] %Titulo del TFG
    
    \rule[0.5ex]{\linewidth}{1pt}\vspace*{-\baselineskip}\vspace{3.2pt}
    \rule[0.5ex]{\linewidth}{2pt}\\
    \vspace{6.5mm}
    {\large Por\\Enrique de la Calle Montilla}
    \vspace{6.5mm}
    {\large\textsc{}}\\ %Nombres de los autores
    \vspace{5mm}
    \includegraphics[width=\textwidth]{logo}\\ %Logo de la UCM
    \vspace{6mm}
    {\large Grado de Ingeniería Informática\\    %Departamento del profesor que es el tutor
    \textsc{Facultad de Informática}}\\ %Facultad para la que se hace
    \vspace{11mm}
    \begin{minipage}{10cm}
    \begin{center}
    Carlos García Sánchez\\
      \textbf{Evaluation and acceleration of Path Tracing algorithm in heterogeneous architectures}\\
      \vspace{2mm}
    \end{center}
    \vspace{4mm}
    
    %Cuadro de presentación del TFG
    \end{minipage}\\
    \vspace{4mm}
    {\large\textsc{Madrid, 2020-2021}} %Fecha de finalización o presentación
    \vspace{12mm}
\end{center}

 
\tableofcontents

%\section{Dummy section}
%\subsection{Dummy subsection}
%\subsubsection{Dummy subsubsection}
%\paragraph{Dummy paragraph}

\chapter*{Resumen}
\markboth{RESUMEN}{RESUMEN}

En este trabajo se explica la implementación de Eleven Renderer, un motor de renderizado gráfico de mi autoría, programado en C++ y CUDA desde cero para este TFG. El código fuente se puede encontrar en el repositorio Github \url{https://github.com/101001000/tfg-pathtracer/}. Cuenta con todas las características necesarias para visualizar una escena 3D con relativa eficiencia y hace uso de la arquitectura paralela que ofrecen los aceleradores gráficos. Además se acompaña de una evaluación junto a la justificación de algunas decisiones tomadas durante el desarrollo. A lo largo del trabajo se explonen recursos en caso de querer ampliar información. Así pues, se espera que este trabajo sirva como breve guía y motivación para futuros programadores gráficos.

En este documento también se incluyen las mejoras visuales y optimizaciones aplicadas, junto a varias estadísticas de rendimiento comparando diferentes parámetros. Finalmente se ha puesto a disposición del lector la documentación necesaria para poder crear escenas 3D y visualizarlas de manera fotorrealista con Eleven Renderer, gracias a la posibilidad de importar de manera manual, escenas desde la gran mayoría de software de edición 3D. 

Palabras clave: Renderizado 3D, Path Tracing, Trazado de Rayos, CUDA, Motor Gráfico, Aceleración por GPU.

\chapter*{Abstract}
\markboth{ABSTRACT}{ABSTRACT}

In this work it's explained the implementation of Eleven Renderer, a rendering engine made by me from scratch, coded in C++ and CUDA library, done for this Bachellor Thesis. The source code can be found in the Github's repo \url{https://github.com/101001000/tfg-pathtracer/}. It has all the features needed for 3D scene visualization with relative efficiency and it take advantage of the parallel architecture which graphics accelerators offers. There's also an evaluation for the implementation and the justification of some decisions taken. Some resources are provided in case the reader want in-depth insight about the topic. That been said, it's expected from this work to be used as a small guide and motivation for future developers and graphics programmers.

In this document is also included some visual upgrades and optimizations, accompanied with benchmarks compairing different parametters. Its also offered to the reader, the documentation needed for creating 3D scenes and render it in a photorrealistic way in Eleven Renderer, thanks to the possibility of manually importing scenes from another 3D editing software.

Keywords: 3D Rendering, Path Tracing, Ray Tracing, CUDA, Graphics Engine, GPU Acceleration.


\chapter*{Agradecimientos}
\markboth{AGRADECIMIENTOS}{AGRADECIMIENTOS}

No hubiera sido capaz de realizar este trabajo sólo sin el apoyo de la gente que me ha rodeado a lo largo de estos años de estudio. En primer lugar quiero dar las gracias a mi madre, que aunque no entienda lo que hago, se asegura de que salga adelante. En segundo lugar a mi padre el cual me ha enseñado que el trabajo no es trabajo si te gusta lo que haces. No puedo olvidarme tampoco de mi hermana Violeta, que tengo que quererla.

Por otro lado quiero dar las gracias a Patricia, que ha tenido que aguantar mis cabezazos contra el teclado cada vez que aparecía un error gráfico, y que junto a Hizán, han leído pacientemente mi trabajo y me han ayudado a expresar mejor el caos que tengo por cabeza.

Además quiero dar las gracias a Jorge que ha escuchado con interés el desarrollo de este trabajo y que ha servido como combustible para mi motivación. A él se suman Dani, Jorge, Guille y Alex, que me han acompañado en la carrera y que sin ellos, los cuatro años de carrera no hubieran sido los mismos.

Por último quiero también dar las gracias a mi tutor Carlos, que me enseñó los fundamentos de la computación en paralelo y nos propuso prácticas relacionadas con el procesamiento de imágenes. Aunque no se lo haya dicho, para mí, ha sido muy importante poder realizar trabajos relacionados con la computación gráfica, la que espero que sea mi futura carrera profesional.


\newpage

\includegraphics[
    width=1.45\textwidth,
    height=2\textwidth,
    align=t,
    smash=br,
    vshift=4cm,    
    hshift=-4.1cm
]{portada}

\scalebox{5}{\color{white}{Eleven Renderer}}


\parindent=0em
\chapter{Capítulo 1}
\pagenumbering{arabic}
\noindent
\texttt{Lorem ipsum dolor sit amet, consectetuer ad}\\

\lipsum[1-20]


















\chapter{Fundamentos del algoritmo de Path Tracing}
	

En este capítulo se procede a dar un esquema básico de los fundamentos de este algoritmo. El objetivo será así describir un motor de renderizado previo a cualquier optimización, que cuenta con las funcionalidades básicas para producir una imagen de una escena tridimensional simple.

	\section{Aproximación de la ecuación de renderizado}
\[
{\displaystyle L_{\text{o}}(\mathbf {x} ,\omega _{\text{o}})=L_{\text{e}}(\mathbf {x} ,\omega _{\text{o}})\ +\int _{\Omega }f_{\text{r}}(\mathbf {x} ,\omega _{\text{i}},\omega _{\text{o}})L_{\text{i}}(\mathbf {x} ,\omega _{\text{i}})(\omega _{\text{i}}\cdot \mathbf {n} )\operatorname {d} \omega _{\text{i}}}
\]

La ecuación de renderizado \cite{kajiya1986rendering} aparece por primera vez en 1986 junto al algoritmo de Path Tracing, siendo este algoritmo una propuesta para resolverla. Es el pilar de la visualización 3d fotorrealista ya que simula de una manera suficientemente precisa la interacción de la luz en una escena tridimensional.

La interpretación de esta ecuación es la siguiente: Para un punto $\mathbf {x}$ del espacio y un ángulo $\omega _{\text{o}}$ desde el cual se observa a dicho punto, cuál es la cantidad de energía lumínica que el observador recibe $L_{\text{o}}$.

El primer término $L_{\text{e}}(\mathbf {x} ,\omega _{\text{o}})$ indica la luz que dicho punto $\mathbf {x}$ emite, así pues se podrán modelar materiales que emitan luz propia y no dependan de energía externa.

El segundo término calcula toda la luz entrante y reflejada a través del ángulo $\omega _{\text{o}}$ por dicho punto $\mathbf {x}$, es por ello que integra todos los ángulos del hemisferio superior. Este segundo término se compone de tres coeficientes:

El primer coeficiente $f_{\text{r}}(\mathbf {x} ,\omega _{\text{i}},\omega _{\text{o}})$ es la función BRDF, la cual es dependiente del material e indica cuánta energía se refleja en dicho punto para las direcciones de entrada $\omega _{\text{i}}$ y salida $\omega _{\text{o}}$.

El segundo coeficiente $L_{\text{i}}(\mathbf {x} ,\omega _{\text{i}})$ hace referencia a toda la energía lumínica entrante de todas las direcciones posibles.

El tercer coeficiente $(\omega _{\text{i}}\cdot \mathbf {n})$ es el producto de la ley del coseno de Lambert \cite{lambert1760jh}, un escalar que atenúa los ángulos menos pronunciados con la normal de la superficie.


	\section{Esquema simplificado del trazado de rayos}
	
Habiendo definido la ecuación de renderizado, el siguiente paso es explicar la implementación hecha de los fundamentos del algoritmo de Path Tracing, ya que posteriormente se irá mejorando paso por paso este algoritmo básico.

En su esencia consiste en trazar rayos o "caminos" desde una cámara virtual a una escena tridimensional, simulando así un modelo simplificado de fotones y sus interacciones con la escena. Tras cada interacción con la escena, estos caminos de fotones tienen una pérdida de energía que se acumulará en cada píxel y definirá el color de este, como si del sensor de una cámara real se tratara.

El primer paso es preparar la escena a renderizar \code{Scene}. Una escena básica se compone de una cámara \code{Camera}, geometrías \code{MeshObject} y materiales \code{Material}.

Las cámaras \code{Camera} consisten en una simulación aproximada de una cámara física real, así pues sus atributos serán: tamaño del sensor (en metros) \code{sensorWidth} y \code{sensorHeight}, distancia focal (en metros) \code{focalLength} y resolución (en píxeles) \code{xRes} e \code{yRes}. 

Las geometrías \code{MeshObject} por otra parte consisten en un conjunto de triángulos \code{Tri}, los cuales a su vez consisten en 3 puntos tridimensionales \code{Vector3 vertices[3]}.

Los materiales \code{Material} definen la manera en la que los fotones interactúan con ellos. En su forma más primitiva consisten en un color base, el cual absorberá ciertas longitudes de onda en mayor o menor medida. Para simplificar las computaciones, no es necesario calcular estas interacciones con todo el espectro electromagnético visible, basta con usar los tres colores primarios aditivos: rojo, verde y azul, así pues un color consiste en un vector \code{Vector3(R,G,B)}.

Habiendo definido estos elementos en la escena, se procederá a transferirlos a la GPU, donde se realizará el resto de computaciones más demandantes. Esto es realizado por la función \code{cudaError\_t renderSetup(Scene* scene)}, la cual a través de las funciones de la API de CUDA \code{cudaMalloc} y \code{cudaMemcpy} copia la información de la escena y lleva la cuenta de la memoria transferida en las variables globales \code{textureMemory, geometryMemory}. La copia de memoria de la CPU a la GPU requiere de un tratado especial para los objetos, requiriendo así una copia profunda de los atributos en formato array o punteros, además de un proceso posterior denominado "pointer binding".

%@todo explain pointerbinding
%@todo repharaphrasing and better organize this


\begin{figure}[H]
    \centering
	\includegraphics[width=0.5\textwidth]{pointerbinding}
	\caption{Pointer Binding}
	\label{fig:label}
\end{figure}

Una vez están todos los componentes necesarios en la GPU, es preciso llamar a un kernel para configurar el motor en la GPU. Este kernel \code{setupKernel} se hace cargo de inicializar los bufferes de píxeles y contador de rayos, inicializar \code{curand} (la librería utilizada para generar números aleatorios en CUDA) y por último asignar a cada objeto el índice de triángulos a l

%@todo improve setup kernel



	\subsection{Trazado de rayo desde la cámara}

Para simular el trazado del rayo desde la cámara hasta la escena, se simula de manera simplificada como funcionaría una cámara estenopéica. Se calcula la posición del sensor por la cual se trazará el rayo a partir de las coordenadas \code{x} e \code{y}. 

Puesto que no se está teniendo en cuenta la rotación de la cámara, la coordenada z del sensor se puede simplificar con la distancia de la cámara hasta el sensor.

\begin{lstlisting}
	
__device__ void calculateCameraRay(int x, int y, Camera& camera, Ray& ray, float r1, float r2) {

    // Relative coordinates for the point where the first ray will be launched
    float dx = camera.position.x + ((float)x) / ((float)camera.xRes) * camera.sensorWidth;
    float dy = camera.position.y + ((float)y) / ((float)camera.yRes) * camera.sensorHeight;

    // Absolute coordinates for the point where the first ray will be launched
    float odx = (-camera.sensorWidth / 2.0) + dx;
    float ody = (-camera.sensorHeight / 2.0) + dy;

    // Random part of the sampling offset so we get antialasing
    float rx = (1.0 / (float)camera.xRes) * (r1 - 0.5) * camera.sensorWidth;
    float ry = (1.0 / (float)camera.yRes) * (r2 - 0.5) * camera.sensorHeight;

    // Sensor point, the point where intersects the ray with the sensor
    float SPx = odx + rx;
    float SPy = ody + ry;
    float SPz = camera.position.z + camera.focalLength;

    // The initial ray is created from the camera position to the sensor point. No rotation is taken into account.
    ray = Ray(camera.position, Vector3(SPx, SPy, SPz) - camera.position);
}

\end{lstlisting}

	\subsection{Intersección triángulo - rayo}
	\label{subsec:triintersection}
	
El cálculo de la intersección de un rayo con un triángulo es una de las operaciones más fundamentales de este algoritmo. Esta operación toma como parámetros un triángulo \code{Tri} y un rayo \code{Ray} y ofrece como resultado si dicho triángulo y rayo intersectan en el espacio y además un objeto \code{Hit} el cual cuenta con información adicional de la intersección.

La información adicional que devuelve esta operación es la siguiente:

\begin{itemize}
	
	\item \code{int hit.objectID}: ID del objecto al que pertenece dicho triángulo.
	
	\item \code{Vector3 hit.position}: Posición en el espacio del punto de intersección entre el rayo y el triángulo.
	
	\item \code{Vector3 hit.normal}: Normal de la superficie, calculada a partir del producto vectorial de dos aristas del triángulo.
	
	\item \code{bool hit.valid}: Validez de una intersección, por defecto falso. Verdadero en caso de haber intersectado correctamente.

Para la implementación se ha hecho uso del algoritmo Fast Minimum Storage Ray/Triangle Intersection\cite{moller1997fast}. En este paper se explica detalladamente el algoritmo de intsersección, mientras este trabajo se limita a implementar dicho algoritmo a partir de una adaptación de la implementación en C que el autor ofrece.


\end{itemize}
	
\begin{lstlisting}
	
__host__ __device__ inline bool hit(Ray& ray, Hit& hit) {

float EPSILON = 0.0000001;

        Vector3 edge1 = vertices[1] - vertices[0];
        Vector3 edge2 = vertices[2] - vertices[0];

        Vector3 pvec = Vector3::cross(ray.direction, edge2);

        float u, v, t, inv_det;

        float det = Vector3::dot(edge1, pvec);

        inv_det = 1.0 / det;

        if (det > -EPSILON && det < EPSILON) return false;

        Vector3 tvec = ray.origin - vertices[0];

        u = Vector3::dot(tvec, pvec) * inv_det;
        if (u < 0.0 || u > 1.0)
            return false;

        Vector3 qvec = Vector3::cross(tvec, edge1);
        v = Vector3::dot(ray.direction, qvec) * inv_det;
        if (v < 0.0 || (u + v) > 1.0)
            return false;

        t = Vector3::dot(edge2, qvec) * inv_det;

        if (t < 0) return false;

        Vector3 geomPosition = ray.origin + ray.direction * t;
		Vector3 geomNormal = Vector3::cross(edge1, edge2).normalized();
		
		hit.normal = geomNormal;
		hit.position = geomPosition;
		hit.valid = true;
		hit.objectID = objectID;

        return true;

\end{lstlisting}

\section{Renderizado progresivo}
		
	Una ventaja de los motores de renderizado más modernos es el renderizado progresivo. Esto implica que las muestras se van acumulando poco a poco a lo largo de la imagen hasta que termina por converger. Esto difiere de los motores de renderizado por CPU tradicionales, que acumulan las muestras en secciones locales y una vez que acumulan las suficientes, pasan a la siguiente sección. Se ha decidido hacer una implementación progresiva con el fin de estar más cerca del estado del arte.
	
	Este tipo de implementación se beneficia de la copia de datos asíncrona de la GPU. Mientras el kernel se ejecuta, un flujo de datos secundario hará la copia del buffer de la GPU en la CPU, pudiendo así actualizar la visualización del resultado varias veces por segundo.

	Este flujo de datos secundario se ha implementado con el tipo de datos \code{cudastream\_t} de la API de CUDA. Han sido necesarios dos flujos, uno denominado \code{kernelStream} y otro denominado \code{bufferStream}. Los kernels de inicialización y renderizado correrán en el primero, mientras que la función que obtiene el buffer, será lanzada en el segundo.
	
	La función que extrae el buffer de píxeles de la GPU a la CPU es la siguente:
	
	\begin{lstlisting}
	cudaError_t getBuffer(float* pixelBuffer, int* pathcountBuffer, int size) {

		cudaStreamCreate(&bufferStream);

		cudaError_t cudaStatus = cudaMemcpyFromSymbolAsync(pixelBuffer, dev_buffer, size * sizeof(float) * 4, 0, cudaMemcpyDeviceToHost, bufferStream);
		if (cudaStatus != cudaSuccess) {
			fprintf(stderr, "returned error code %d after launching addKernel!\n", cudaStatus);
		}

		cudaStatus = cudaMemcpyFromSymbolAsync(pathcountBuffer, dev_pathcount, size * sizeof(unsigned int), 0, cudaMemcpyDeviceToHost, bufferStream);
		if (cudaStatus != cudaSuccess) {
			fprintf(stderr, "returned error code %d after launching addKernel!\n", cudaStatus);
		}

		return cudaStatus;
	}
	\end{lstlisting}
	
	Hace uso de la función \code{cudaMemcpyFromSymbolAsync} para realizar la copia asíncrona antes mencionada. También se hace copia del buffer de la suma de rayos emitidos por píxel con el fin de realizar métricas de eficiencia.
	
	%@todo analisis de movimiento de datos de memoria.


	











\chapter{Mejoras visuales}
				
En el capítulo anterior se definió la implementación básica para visualizar una escena simple. Así pues, este capítulo explica las mejoras visuales más comunes utilizadas en los motores gráficos de producción, así como su implementación en Eleven Renderer y su evaluación.
	
\section{Desenfoque: Modelo de lente fina}
	
Hasta ahora se ha estado utilizando un modelo de cámara ideal denominado cámara estenopeica. Esta cámara tiene la particularidad de tener un enfoque perfecto siempre, siendo una propiedad indeseada en un motor de renderizado fotorrealista, esto se debe a que a diferencia de las cámaras estenopeicas, , las cámaras reales incluyen lentes en su estructura que desvían los rayos de luz gracias a la difracción del cristal, enfocando a determinada distancia, y desenfocando el resto de la escena. El hecho de poder enfocar a una distancia determinada permite hacer énfasis en un sujeto de la escena y desenfocar el resto. Este efecto se conoce como \emph{Bokeh} y es muy deseado en un motor de renderizado, puesto que es un recurso cinematográfico muy atractivo visualmente.

Para solventar el problema del enfoque perfecto se hace uso de un modelo de cámara denominado modelo de lente fina. Este modelo es una simplificación de lo que sería una simulación física de unas lentes reales. Al simplificar los cálculos, se pierden artefactos y desperfectos deseados como la aberración cromática o la distorsión de lentes, pero a cambio se obtiene la simplicidad de implementación.

Para activar este efecto, es necesario compilar con la constante \code{BOKEH} definida. Esto desbloqueará la parte del código que hace el cálculo del desenfoque, ubicado en \autoref{cod:bokeh} 
	
Este nuevo método añade a la clase \code{Camera} dos nuevas variables, por un lado \code{focusDistance} y por otro lado \code{aperture}. La primera define la distancia a la que se encuentra el plano de enfoque, y la segunda, la apertura en f-stops del iris de la cámara. 

El procedimiento para calcular los rayos emitidos por el nuevo modelo de cámara se muestra de manera visual en la \autoref{cod:bokeh} y es el siguiente:

\begin{enumerate}
		
\item Se calcula el rayo original del método anterior, desde la cámara hasta el sensor.

\item Se calcula la intersección de dicho rayo con el plano de enfoque, situado a la distancia \code{focusDistance}. La intersección se denomina \code{focusPoint}.
	
\item En vez de emitir el rayo desde el punto de la cámara, se elige un punto aleatorio en el iris \code{iRP} y se emite un rayo desde ahí hasta el punto de enfoque \code{focusPoint}. Este nuevo rayo será un rayo bajo el modelo de cámara de lente fina. Los elementos situados a la distancia de enfoque \code{focusDistance} serán más nítidos que aquellos que no lo estén.
	
\end{enumerate}

\begin{minipage}[c]{0.95\textwidth}
\begin{lstlisting}[label={cod:bokeh}, caption={Código de desenfoque.}]
	
	#if BOKEH
	
    float rIPx, rIPy;

    // The diameter of the camera iris
    float diameter = camera.focalLength / camera.aperture;

    // Total length from the camera to the focus plane
    float l = camera.focusDistance + camera.focalLength;

    // The point from the initial ray which is actually in focus
    Vector3 focusPoint = ray.origin + ray.direction * l;

    // Sampling for the iris of the camera
    uniformCircleSampling(r3, r4, r5, rIPx, rIPy);

    rIPx *= diameter * 0.5;
    rIPy *= diameter * 0.5;

    Vector3 orig = camera.position + Vector3(rIPx, rIPy, 0);

    //Blurred ray
    ray = Ray(orig , focusPoint - orig);

	#endif 

\end{lstlisting}
\end{minipage}

\begin{figure}[H]
	\centering
	\includegraphics[width=0.7\textwidth]{blurring}
	\caption{Esquema cámara modelo de lente fina.}
	\label{fig:thinlensecamera}
\end{figure}

El resultado final es un desenfoque configurable a partir de la distancia de enfoque. Es posible, también, aumentar la cantidad de desenfoque reduciendo el número de f-stops, que consecuentemente aumentará la apertura del iris. La \autoref{fig:focusboxes} muestra un ejemplo de la modificación de distancia de enfoque mencionada.

\begin{figure}[H]
	\centering
  \begin{minipage}[b]{0.3\textwidth}
	\includegraphics[width=\textwidth]{nofocus}
	\caption{Desenfoque desactivado.}
  \end{minipage}
  \hfill
  \begin{minipage}[b]{0.3\textwidth}
	\includegraphics[width=\textwidth]{focusnear}
	\caption{Distancia de enfoque 10m.}
  \end{minipage}
  	\hfill
  \begin{minipage}[b]{0.3\textwidth}
	\includegraphics[width=\textwidth]{focusfar}
	\caption{Distancia de enfoque 30m.}
  \end{minipage}
\caption{Comparación de distintas distancias de enfoque.}
\label{fig:focusboxes}
\end{figure}
	
		
\section{Texturas}
	
El uso de colores planos en los materiales limita la capacidad de imitación de la realidad. Un recurso esencial para romper esta limitación es el uso de texturas. Una textura consiste en una matriz bidimensional de valores en punto flotante. En la implementación se ha añadido una clase \code{Texture} que contiene dos valores enteros \code{width} y \code{height}, que indican la altura y anchura de la matriz de datos \code{data}. La carga se hace a partir de un constructor que toma como parámetro la dirección de una imagen en formato .bmp y dividirá el valor de cada canal y cada pixel por 256 con el fin de limitar el rango de valores en [0,1). Los valores \code{width} y \code{height} son extraídos de la cabecera. El constructor admite un parámetro opcional \code{colorSpace}, que determina si es necesario convertir la imagen cargada de espacio de color. Por el momento, solo se soportan dos espacios de color, el espacio Lineal y el espacio sRGB, que son los más comunes.

Para cambiar de Lineal a sRGB se eleva el valor a 2.2, que es el valor de corrección de gamma. Para cambiar de sRGB a Lineal, se eleva el valor a $\frac{1}{2.2}$.
	
	%@todo add image of a color texture
	
\subsection{Mapeo}

Puesto que cada textura puede tener distintas resoluciones, es bastante útil definir un sistema de coordenadas relativas a la altura y anchura de una textura. Este sistema se conoce como sistema de coordenadas \code{u,v}. Ambas coordenadas \code{u} y \code{v} son valores de punto flotante comprendidos entre [0,1]. \code{u} indica la coordenada horizontal mientras que \code{v} indica la coordenada vertical. Estas coordenadas son diferentes a las coordenadas \code{u,v} explicadas en \hyperref[subsec:triintersection]{Intersección triángulo - rayo} aunque tengan el mismo nombre.

También resulta útil definir dos parámetros de transformación para las texturas. Estos son \code{Tile} y \code{Offset}. El primero indica el inverso de la escala de la textura; útil, por ejemplo, si se busca que una textura se repita cierto número de veces. El segundo es el desplazamientos de esta en dos ejes. Dicha transformación queda contemplada en \autoref{cod:getval}

\begin{minipage}[c]{0.95\textwidth}
\begin{lstlisting}[label={cod:getval}, caption={Código para obtener valor de una textura.}]
	
	__host__ __device__ Vector3 getValueFromCoordinates(int x, int y) {
	
        Vector3 pixel;

        // Offset and tiling tranforms
        x = (int)(xTile * (x + xOffset)) % width;
        y = (int)(yTile * (y + yOffset)) % height;

        pixel.x = data[(3 * (y * width + x) + 0)];
        pixel.y = data[(3 * (y * width + x) + 1)];
        pixel.z = data[(3 * (y * width + x) + 2)];

        return pixel;
    }
\end{lstlisting}
\end{minipage}

\subsection{IBL (Image Based Lightning)}
	
Hasta ahora, la luz que llegaba a la escena desde el fondo era una radiación homogénea determinada por un color, sin embargo, es posible utilizar imágenes para iluminar la escena.
	
La iluminación basada en imagen ha sido uno de los elementos más relevantes en las técnicas para el renderizado fotorrealista. Se utiliza ampliamente en la industria cinematográfica debido a la complejidad visual que aporta a una escena 3D porque permite captar la iluminación de entornos reales y posteriormente añadirla en escenas digitales. El hecho de poder trasladar la iluminación a un escenario virtual facilita la composición de modelos tridimensionales en películas y series de televisión, donde es necesario juntar una grabación real con un elemento generado por ordenador.

Esta técnica se basa principalmente en usar una fotografía de 360 grados como fuentes de luz formando una esfera alrededor de la escena. Estas fotografías son conocidas como HDRI (del inglés High Dynamic Range Image). A diferencia de las imágenes tradicionales, las cuales normalmente tienen 8 bits de resolución por canal de color, las imágenes HDRI cuentan con valores de punto flotante. El uso de las imágenes HDRI no se limita a la visualización de estas en pantallas de 8 bits de resolución por color como la mayoría de imágenes, sino que el valor de cada píxel será utilizado para realizar las operaciones pertinentes para iluminar la escena. Un ejemplo de este tipo de iluminación se aprecia en la \autoref{fig:hdriexample}

		
\begin{figure}[H]
	\label{fig:hdriexample}
    \centering
	\includegraphics[width=0.7\textwidth]{hdriexample}
	\caption{Escena iluminada con una imagen HDRI de estudio, nótese como la luz superior incide directamente en la esfera acorde a la dirección.}
\end{figure}

La implementación de esta técnica en el motor de render viene dada por el uso de una imagen en formato .hdr (imágenes en punto flotante). Cada píxel de esta imagen se interpreta como una pequeña fuente de luz direccional en el infinito, orientada hacia el centro de la escena. Así pues, los rayos que no interseccionan en la escena con ningún elemento se considera que interseccionan en el infinito con el \code{HDRI}. Por esta razón, al detectar que un rayo sale de la escena, se obtiene su dirección, y esta dirección se traduce en las coordenadas polares de la imagen \code{HDRI}. Una vez se tienen las coordenadas polares, es posible obtener el valor lumínico del píxel al que apunta dicho rayo. Este será el valor lumínico que se utilice para dicho camino.
	

Debido a la naturaleza esférica de los mapas de entorno, resulta útil añadir dos funciones que transforman coordenadas esféricas en coordenadas \code{u,v}. Estas dos funciones son \code{sphericalMapping}, definida en \autoref{cod:sphericalmapping} y su inversa \code{reverseSphericalMapping}, definida en \autoref{cod:reversesphericalmapping}. La primera devuelve las coordenadas \code{u,v} para un punto situado en la superficie de una esfera de radio arbitrario mientras que la segunda calcula la posición de un punto en la superficie de una esfera de radio unitario, dadas dos coordenadas \code{u,v}.

\label{sphericalmapping}

\begin{minipage}[c]{0.95\textwidth}
\begin{lstlisting}[label={cod:sphericalmapping}, caption={Código para calcular las coordenadas u,v a través de un vector en una esfera de radio determinado.}]
	
    __host__ __device__ static inline void sphericalMapping(Vector3 origin, Vector3 point, float radius, float& u, float& v) {

        // Point is normalized to radius 1 sphere
        Vector3 p = (point - origin) / radius;

        float theta = acos(-p.y);
        float phi = atan2(-p.z, p.x) + PI;

        u = phi / (2 * PI);
        v = theta / PI;

        limitUV(u,v);
    }
	
\end{lstlisting}
\end{minipage}

\begin{minipage}[c]{0.95\textwidth}
\begin{lstlisting}[label={cod:reversesphericalmapping}, caption={Función inversa al mapeo esférico.}]
		
	__host__ __device__ static inline Vector3 reverseSphericalMapping(float u, float v) {

        float phi = u * 2 * PI;
        float theta = v * PI;

        float px = cos(phi - PI);
        float py = -cos(theta);
        float pz = -sin(phi - PI);

        float a = sqrt(1 - py * py);

        return Vector3(a * px, py, a * pz);
    }
	
\end{lstlisting}
\end{minipage}


\subsection{Filtrado}
			
Las texturas cuentan con valores discretos y resoluciones limitadas. Esto provoca que la imagen se pixele cuando se muestra cercana a la cámara. Una solución adoptada de manera general en muchos ámbitos es la interpolación de los píxeles vecinos. Actualmente, muchos visualizadores de imágenes no muestran los píxeles naturales, sino una versión modificada de estos. Esto se conoce como filtrado. En la implementación se ha usado un filtrado lineal conocido como interpolación bilineal. Este tipo de filtrado es sencillo de entender e implementar. El valor del punto \code{x} comprendido entre los cuatro píxeles más cercanos \code{v1-v4} es la suma ponderada de la distancia del punto a cada píxel en cada dimensión.
	
\begin{figure}[H]
	\label{fig:bilinear}
	\centering
	\includegraphics[width=0.3\textwidth]{bilinear}
	\caption{Interpolación bilineal.}
\end{figure}
	
\begin{minipage}[c]{0.95\textwidth}
\begin{lstlisting}[label={cod:bilinear}, caption={Código interpolación bilineal.}]
	
	__host__ __device__ Vector3 getValueBilinear(float u, float v) {
        
        float x = u * width;
        float y = v * height;

        float t1x = floor(x);
        float t1y = floor(y);

        float t2x = t1x + 1;
        float t2y = t1y + 1;

		// Weights per dimension
        float a = (x - t1x) / (t2x - t1x);
        float b = (y - t1y) / (t2y - t1y);

		// Rounded neighbour values
        Vector3 v1 = getValueFromCoordinates(t1x, t1y);
        Vector3 v2 = getValueFromCoordinates(t2x, t1y);
        Vector3 v3 = getValueFromCoordinates(t1x, t2y);
        Vector3 v4 = getValueFromCoordinates(t2x, t2y);

		// Linear interpolation
        return lerp(lerp(v1, v2, a), lerp(v3, v4, a), b);
	}
\end{lstlisting}
\end{minipage}	

Este método introduce cuatro lecturas del valor de la textura en vez de la única lectura sin usar técnicas de filtrado, de manera que resulta conveniente analizar el posible impacto en la eficiencia. Para comprobar esto se ha renderizado la misma escena con y sin filtrado en la \autoref{fig:filtercomparaison}. 
	
El render sin filtrado es un 0.8\% más rápido, una penalización poco sustancial, por lo que se utilizará el filtrado bilinear de manera predeterminada. Se esperaría que esta penalización fuera mayor debido a la multiplicación de accesos a memoria, pero como se explica en el \autoref{chap:evaluation}, la mayor carga de trabajo reside en las intersecciones con los objetos en la escena.

\begin{figure}[H]
\centering
\begin{tikzpicture}
       \begin{axis}[
           symbolic x coords={Bilinear filtering, No filtering},
           xtick=data,
		/pgf/number format/.cd,
		1000 sep={},
		scaled y ticks=false,
		ylabel=KPaths/s,
		ylabel style={yshift=0.5cm},
         ]
           \addplot[ybar,fill=blue] coordinates {
               (Bilinear filtering,   15965)
               (No filtering,  16093)
           };
       \end{axis}
\end{tikzpicture}
\caption{Comparación de eficiencia para filtrado / no filtrado.}
\label{fig:filtercomparaison}
\end{figure}
	
\subsection{Mapas de normales}
		
A nivel artístico resulta muy útil definir la normal para cada intersección en un triángulo de manera arbitraria, aporta control sobre la dirección en la que la luz incide en la superficie; además de que permite dar mayor complejidad y detalle a las geometrías contar con la penalización que implica hacer uso de triángulos adicionales. Un ejemplo de esto es la \autoref{fig:normalmap}, donde se puede ver como una simple textura puede simular relieve en la superficie de una esfera a partir de la textura en la \autoref{fig:normalmaptexture}. 
	
		
\begin{figure}[H]
    \centering
	\includegraphics[width=0.6\textwidth]{normalmap}
	\caption{Esfera con mapa de normales.}
	\label{fig:normalmap}
\end{figure}

\begin{figure}[H]
    \centering
	\includegraphics[width=0.3\textwidth]{normalmaptexture}
	\caption{Textura para mapa de normales.}
	\label{fig:normalmaptexture}
\end{figure}


	
La mayor parte de la información acerca de los mapas de normales se ha obtenido en LearnOpenGL.com \cite{learnopengl}, donde se explican las nociones teóricas sobre estos.
	
Estas normales se suministran a través de texturas de una forma similar a la mencionada anteriormente. La textura que carga la información del mapa de normales cuenta con tres canales de color. Cada canal se utiliza para definir la coordenada de la normal, siendo el canal rojo la coordenada \code{x}, el canal azul la coordenada \code{y} y el canal verde la coordenada \code{z}.

La conversión de un color a una normal se hace con el siguiente código: \code{localNormal = (color * 2) - 1;} ya que las coordenadas de la normal se encuentran en el rango [-1, 1] y los valores de color se encuentran en el rango [0,1).
	
Hasta ahora, en ningún momento se ha tenido en cuenta el espacio en el que se encuentran estas normales. Los vectores suministrados por los mapas de normales son locales en el espacio tangencial. El espacio tangencial es un espacio formado por la normal calculada a partir del triángulo, la tangente y la bitangente, tres vectores aproximadamente ortogonales. La representación visual de estos tres vectores que forman el espacio tangencial se muestra en la \autoref{fig:tangentspace}.

\begin{figure}[H]
    \centering
	\includegraphics[width=0.4\textwidth]{tangentspace}
	\caption{Espacio tangencial y su proyección en una textura.}
	\label{fig:tangentspace}
\end{figure}
	
Las normales que se han utilizado hasta ahora para los cálculos de iluminación son normales globales (world normals), en la base ortonormal. La conversión de un vector en el espacio tangencial al global se realiza de la siguiente manera: \code{worldNormal = (localNormal.x * tangent - localNormal.y * bitangent + localNormal.z * normal).normalized()}
	
El cálculo de la tangente y bitangente a partir de la normal no es un cálculo trivial, hay infinitos vectores ortogonales que pueden formar esta base. Morten Mikkelsen \cite{mikkelsen2008simulation} propone un método estandarizado para calcular este espacio tangencial y además aporta una implementación en C en su repositorio \cite{mikktspace}. Gran parte de los programas de diseño 3D utilizan este método o uno compatible. Por esa razón se ha decidido hacer uso de él. El método en sí se encarga de cada caso específico, como por ejemplo polígonos degenerados, pero en su esencia calcula la tangente en dirección al incremento de la coordenada \code{u} en el triángulo y la bitagente en dirección de la coordenada \code{v}. Este proceso se explica de manera más detallada en un árticulo de LearnOpenGL \cite{learnopengl}.

Para hacer uso de la implementación MikkTSpace tan solo hay que definir un archivo adicional que enlace las funciones de callback, que solicitarán parámetros para el cálculo de las tangentes tales como las posiciones de los vértices, el número de triángulos etc.

	
\section{Smooth Shading}
	
En \hyperref[subsec:triintersection]{Intersección triángulo - rayo} se explica cómo las normales son calculadas a partir de la superficie que forma el triángulo. Para modelos con poca cantidad de triángulos, este método de cálculo de la normal de la superficie puede resultar insuficiente.
	
Un arreglo sencillo consiste en aplicar el método conocido como Smooth Shading. 
	
Para este método es necesario precalcular una normal por cada vértice, algo que hacen casi todos los programas de diseño 3D. En el caso de los ficheros .obj, estas son definidas con el prefijo "vn". Una vez se cuenta con dichas normales, es necesario interpolarlas. 

Por ello, en la función \code{hit()} de \code{Tri} se añade el código determinado en \autoref{cod:smoothshading}

\begin{minipage}[c]{0.95\textwidth}
\begin{lstlisting}[label={cod:smoothshading}, caption={Código interpolación de normales.}]
			
	#if SMOOTH_SHADING 

    Vector3 shadingNormal  = normals[0] + (normals[1] - normals[0]) * u + (normals[2] - normals[0]) * v;
    Vector3 shadingTangent = tangents[0] + (tangents[1] - tangents[0]) * u + (tangents[2] - tangents[0]) * v;

\end{lstlisting}
\end{minipage}
	
Tanto la normal y la tangente son interpoladas a partir de las coordenadas baricéntricas del triángulo \code{u,v}, las cuales indican la distancia a cada vértice, como se muestra en la \autoref{fig:smoothshading}. El resultado de implementar "Smooth Shading" queda reflejado en las figuras \autoref{fig:smoothshadingoff} y \autoref{fig:smoothshadingon}, donde la primera es un ejemplo de esta característica desactivada y la segunda es un ejemplo de la característica activada.
	
\begin{figure}[H]
    \centering
	\includegraphics[width=0.3\textwidth]{smoothshading}
	\caption{Interpolación de normales.}
	\label{fig:smoothshading}
\end{figure}

\begin{figure}[H]
		\centering
		\begin{minipage}[b]{0.4\textwidth}
		\includegraphics[width=\textwidth]{smoothshadingoff}
		\caption{Smooth Shading desactivado.}
		\label{fig:smoothshadingoff}
	  \end{minipage}
	  \hfill
	  \begin{minipage}[b]{0.4\textwidth}
		\includegraphics[width=\textwidth]{smoothshadingon}
		\caption{Smooth Shading activado.}
		\label{fig:smoothshadingon}
	  \end{minipage}
	  	\hfill
\end{figure}


Al implementar esta mejora pueden aparecer artefactos conocidos como "Shadow Terminator Arctifact". El uso de una normal interpolada implica que se está haciendo uso de una normal que no coincide con la posición de la superficie. Esta discrepancia crea artefactos visuales presentes hasta en las implementaciones más actualizadas de Blender Cycles, véase por ejemplo la \autoref{fig:shadowterminatorcycles}. Mauricio Vives en \cite{shadowterminatorrepo} propone una solución a este problema y es computar una nueva posición de la superficie a partir de la proyección de la antigua posición al plano que define cada vértice y su normal. Estas tres proyecciones son interpoladas de la misma manera que con las normales, haciendo uso de las coordenadas baricéntricas.
	
\autoref{cod:shadowterminator} muestra el código adaptado a Eleven Renderer.

\begin{minipage}[c]{0.95\textwidth}
\begin{lstlisting}[label={cod:shadowterminator}, caption={Código Shadow Terminator.}]
	
	Vector3 p0 = projectOnPlane(geomPosition, vertices[0], normals[0]);
    Vector3 p1 = projectOnPlane(geomPosition, vertices[1], normals[1]);
    Vector3 p2 = projectOnPlane(geomPosition, vertices[2], normals[2]);

    Vector3 shadingPosition = p0 + (p1 - p0) * u + (p2 - p0) * v;

    bool convex = Vector3::dot(shadingPosition - geomPosition, shadingNormal) > 0.0f;
	
	hit.position = convex ? shadingPosition : geomPosition;
	
\end{lstlisting}
\end{minipage}

El resultado de la implementación en Eleven Renderer se muestra en la \autoref{fig:shadowterminatoreleven}

\begin{figure}[H]
	\centering
	  \begin{subfigure}[b]{0.4\textwidth}
		\includegraphics[width=\textwidth]{shadowterminatorcycles}
		\caption{Escena de esfera en Cycles.}
		\label{fig:shadowterminatorcycles}
	  \end{subfigure}
	 \hfill
	  \begin{subfigure}[b]{0.4\textwidth}
		\includegraphics[width=\textwidth]{shadowterminatoreleven}
		\caption{Escena de esfera en Eleven Renderer.}
		\label{fig:shadowterminatoreleven}
	  \end{subfigure}
	  \caption{Artefactos visuales y solución debidos a Shadow Termination.}
      \label{fig:shadowterminator}
	 \hfill
\end{figure}
	
\section{Sombreado BRDF}
	
El sombreado (shading) es el proceso por el cual se asigna un valor de pérdida de energía para un rayo que intersecciona con un punto. Este proceso simula el proceso natural de la interacción entre un haz de fotones y una superficie, dónde parte de los fotones son absorbidos dependiendo de distintos factores como la longitud de onda de los fotones, el tipo de superficie, el tipo de material (los metales reflejarán de manera menos aleatoria que los materiales dieléctricos). La simulación de estas interacciones y simplificación en una función es una ciencia conocida como PBR, o Physically Based Rendering, donde se trata de imitar de manera fiel la realidad física. 
		
En la ecuación de renderizado se encuentra un término denominado BRDF $f_{\text{r}}(\mathbf {x} ,\omega _{\text{i}},\omega _{\text{o}})$. Hasta ahora ha sido ignorado y simplificado como una superficie Lambertiana perfectamente difusa, es decir que se ha tratado la luz independientemente del ángulo de incidencia $\omega _{\text{i}}$ y del ángulo de reflexión del rayo $\omega _{\text{o}}$. En el momento en el que se tienen en consideración estos dos ángulos, se pueden configurar funciones complejas que determinen distintos coeficientes para distintos ángulos. Una función BRDF basada en comportamientos físicos reales como los mencionados anteriormente ofrecerá un resultado acercado a la realidad. 
	
Disney propone una función BRDF conocida como Disney Principled BRDF\cite{burley2012physically}. Este modelo ha sido ideado por Disney con el fin de adaptar los parámetros del modelo de sombreado a unos parámetros amigables para los artistas. Con esto se pretende dejar de lado los antiguos modelos de sombreado poco intuitivos. Esta función de sombreado es utilizada también por Blender Cycles como modelo predeterminado.
	
El término \emph{material} y \emph{BRDF} están fuertemente ligados. Un material es un conjunto de valores para estos atributos.
		
Así pues se procede a explicar estos atributos sintetizados:

\begin{itemize}

	\item Albedo: Color base.
	\item Emission: Energía lumínica añadida, el término $L_{\text{e}}(\mathbf {x} ,\omega _{\text{o}})$ de la ecuación de renderizado.
	\item Roughness: Rugosidad de una superficie.
	\item Metallic: Como de metálica es una superficie, cuanto menor sea la cantidad, más dieléctrico el material.
	\item Clearcoat: Capa de barniz superior.
	\item ClearcoatGloss: Lo pulida que está dicha capa de barniz.
	\item Anisotropic: Distorsión de la luz propia de los metales.
	\item Specular: Cantidad de reflejo especular.
	\item SpecularTint: Color de dicho reflejo.
	\item Sheen: Cantidad de luz en los bordes del material.
	\item SheenTint: Color de la capa de luz anterior.
	\item Subsurface: Sub Surface Scattering falso simulado.
	
\end{itemize}

En Eleven Renderer se ha hecho uso de una implementación ya existente \cite{knightcrawler25} de las ecuaciones del shader de Disney adaptada sin utilizar la transmisión. El propio Disney tiene en Github implementaciones de este shader disponibles \cite{disneyrepo}. Se compara en la \autoref{fig:materialtest} tres parámetros distintos en Eleven Renderer. La primera hilera muestra distintos valores de \emph{albedo}, la segunda muestra la variación de \emph{rougness} incrementando de izquierda a derecha, siendo los materiales de la derecha más lisos. En la tercera hilera se muestra una comparativa del parámetro \emph{metallic}, siendo los materiales de la derecha más metálicos.

Gracias a la implementación previa de texturas es posible además definir estos valores de manera arbitraria para distintas partes del objeto.

\begin{figure}[H]
	\centering
	\includegraphics[width=0.5\textwidth]{materialtest}
	\caption{Comparación de los atributos albedo, roughness y metallic.}
	\label{fig:materialtest}
\end{figure}
	

\chapter{Capítulo 4}
\noindent
\lipsum[1-20]
\chapter{Evaluación}
\label{chap:evaluation}
\label{chap:5}

\section{Evaluación de mejoras}

\subsection{Evaluación BVH}
	
En el \autoref{chap:4} se ha hablado de la implementación de árboles BVH, pero no se ha mostrado ninguna comparativa sobre que ventajas ofrece frente al método ingenuo. Por esta razón, se han evaluado tres escenas, de número de triángulos creciente, sin y con la mejora. El resultado ha sido el esperado: para escenas de muchos triángulos, la diferencia es órdenes de magnitud más eficiente. La complejidad del método ingenuo crece linealmente con el número de triángulos, de manera que deja de ser viable utilizarla en el momento en el que las geometrías se complican.\\

\begin{tabular}{ |p{3cm}||p{3cm}|p{3cm}|p{3cm}| }
	 \hline
	 \multicolumn{4}{|c|}{Accelerator structure BVH comparaison} \\
	 \hline
	 Model Name&No BVH&BVH&Speedup\\
	 \hline
	 Suzanne   &8357 kPath/s&13973 kPath/s&167.20\%\\
	 Stanford Bunny &1806 kPath/s&13960 kPath/s&772.97\%\\
	 Stanford Dragon &14 kPath/s &11507 kPath/s&82192.85\%\\
	 \hline
\end{tabular}

\subsection{Selección de parámetros BVH}
	
Finalmente es necesario elegir los parámetros para \code{BVH\_DEPTH} y \code{SAH\_BINS}. Puesto que existen varios tipos de escenas con distintas geometrías, se va a dejar de lado cualquier tipo de enfoque analítico y se va a hacer uso de medidas en casos reales. Se procede a realizar una evaluación con distintos parámetros.
	
En el caso de la selección de \code{SAH\_BINS}, Indigo Wald propone 16 como límite \cite{wald2007fast} debido a la insustancial mejora de rendimiento para valores mayores. Esto se ha podido comprobar directamente en Eleven Renderer donde se ha llevado a cabo la construcción y recorrido de BVH para la escena "Stanford Dragon" variando la cantidad de contenedores \autoref{fig:sahbins}. Se comprueba que la velocidad de recorrido se estanca para los valores 12-16, mientras que el tiempo de generación permanece lineal. Eleven Render utiliza de manera predeterminada \code{SAH\_BINS = 14} tras el análisis realizado.

\begin{figure}[H]
\centering
\begin{tikzpicture}

\begin{axis}[
    axis y line = right,
    xlabel = \(SAH bins\),
    ylabel = {\(\textcolor{blue}{BVH building time (s)}\)},
	legend style={at={(1,0.1)},anchor=south east},
	scaled y ticks=false
]

\addplot[smooth, blue]
coordinates{(4,4.267) (6, 4.418) (8, 4.628) (10, 5.015) (12, 5.440) (14, 5.782) (16, 6.309)};
\addlegendentry{\(build\)}
\end{axis}

\begin{axis}[
    axis y line = left,
	axis x line = none,
    xlabel = \(SAH bins\),
    ylabel = {\(\textcolor{red}{KPaths/s}\)},
	ylabel style={yshift=0.5cm},
	legend style={at={(1,0)},anchor=south east},
	scaled y ticks=false
]

\addplot [smooth, red]
coordinates{(4,8310) (6, 12012) (8, 12728) (10, 13100) (12, 13046) (14, 13501) (16, 13458)};
\addlegendentry{\(transversal\)}

\end{axis}
\end{tikzpicture}
\caption{Comparación de tiempos de construcción y eficiencia de recorrido para distinto número de contenedores}
\label{fig:sahbins}
\end{figure}


\section{Evaluación Nsight Compute}

Para llevar a cabo la evaluación del algoritmo se ha hecho uso de la herramienta NVIDIA Nsight Compute. Esta herramienta proporciona un análisis completo sobre la ejecución de un kernel, así como información relevante que puede dar pistas de dónde están los cuellos de botella y que partes conviene optimizar.

Tras crear un proyecto y acceder al perfil de \code{renderingKernel}, el primer dato de relevancia que se ha buscado ha sido que parte del código es la más ejecutada. Conociendo este detalle se puede evitar el esfuerzo de optimizar partes poco relevantes. Para conocer este dato es necesario acceder a la pestaña \code{Source} de la aplicación. En esta pestaña se visualiza el código junto a las instrucciones pptx y a la derecha un mapa de calor que indica en qué partes del código frecuentan más los filtros seleccionados en la pestaña \code{Navigation}.

Se ha seleccionado el filtro de instrucciones ejecutadas donde se han podido localizar dos principales puntos calientes. El primero de los dos es el acceso a los hijos en los nodos del árbol BVH. El segundo es en las funciones de mínimo y máximo. Esto es de esperar ya que son funciones aritméticas que se utilizan en el cálculo de la intersección del rayo con los nodos del árbol.

\begin{figure}[H]
    \centering
	\includegraphics[width=0.5\textwidth]{instructionsexecutedbvh}
	\caption{Nº instrucciones ejecutadas funciones leftChild, rightChild}
	\label{fig:label}
\end{figure}

\begin{figure}[H]
    \centering
	\includegraphics[width=0.5\textwidth]{instructionsexecutedminmax}
	\caption{Nº instrucciones ejecutadas funciones minf/maxf}
	\label{fig:label}
\end{figure}

Estos resultados muestran que la mayoría de las instrucciones ejecutadas se encuentran en el recorrido de los árboles BVH. No es de extrañar que la mayoría de los esfuerzos de optimización del trazado de rayos en el estado del arte residan en estructuras de aceleración más óptimas y compactas. 

NVIDIA Nsight Compute también ofrece advertencias de que partes pueden resultar problemáticas o subóptimas en una arquitectura de GPU en el panel \code{Details}. La ejecución de Eleven Renderer provoca el aviso de un error común en arquitecturas paralelas y es el acceso no secuencial de la memoria. Este fenómeno ocurre cuando los hilos de un wrap no acceden a posiciones contiguas, y en el caso de esta evaluación la advertencia redirige a la parte del código que accede a los hijos del árbol BVH. Esto es de esperar puesto que las estructuras de los árboles no ofrecen buena secuencialidad.

\begin{figure}[H]
    \centering
	\includegraphics[width=0.7\textwidth]{memoryaccess}
	\caption{Advertencias de acceso no secuencial}
	\label{fig:label}
\end{figure}


\subsection{Análisis Roofline}
	
El modelo Roofline es utilizado en el análisis de eficiencia de aplicaciones de altas prestaciones. Es un modelo que simplifica la visión del hardware y software mostrando un posible techo de eficiencia. NVIDIA Nsight Compute ofrece este análisis para poder analizar posibles deficiencias y optimizaciones.

La línea superior es una cota superior de 29.23 TFLOPS

El análisis de ejecución de Eleven Renderer para precisión simple \autoref{fig:roofline} ha resultado en 0.36 FLOP/byte de intensidad aritmética y 124.47 GFLOPS de eficiencia, mientras que el techo para una intensidad aritmética de 0.36 FLOP/byte está en 330.65 GFLOPS. Esto indica que el algoritmo está corriendo a un 37.6\% de su capacidad máxima teórica según este modelo siempre y cuando la intensidad aritmética no varíe. Este resultado es razonable, indica que no se está infrautilizando el acelerador gráfico, además indica que el algoritmo tiene una gran dependencia de memoria al encontrarse el punto ubicado a la izquierda.

Si se quisiera optimizar más aún este motor, un buen camino sería romper esta dependencia.

\begin{figure}[H]
    \centering
	\includegraphics[width=0.9\textwidth]{roofline}
	\caption{Análisis Roofline}
	\label{fig:roofline}
\end{figure}



\chapter{Conclusiones}
\label{chap:6}
	
Desarrollar software con el fin de ser ejecutado en aceleradores gráficos puede ser una tarea relativamente sencilla en una primera instancia. El verdadero interés es conseguir el mayor rendimiento y exprimir estas arquitecturas, teniendo en cuenta sus limitaciones y sus puntos fuertes. 

Haciendo referencia al desarrollo de un motor de renderizado fotorrealista, han habido grandes desafíos. En primer lugar, el renderizado gráfico cuenta con un gran trasfondo teórico y matemático del que conviene conocer para poder explotar a fondo cualquier optimización. Este trabajo ha expuesto solo la implementación y la evaluación de este software, sin hacer demasiado hincapié en las bases teóricas. Como se mencionaba en la introducción, Physically based rendering: From theory to implementation \cite{pharr2016physically} es un excelente recurso para esto, denominado por la comunidad como "La biblia del renderizado PBR".

Por otro lado, la programación de Eleven Renderer ha sido mucho más tediosa de lo esperada. Mientras que mejoras como la construcción de estructuras de aceleración a primera vista podrían imponer por su complejidad, han resultado ser más sencillas de implementar que mejoras más sutiles y simples. Esto es debido a la falta de opciones de depuración en GPU. Mientras que el desarrollo en CPU cuenta con una infinidad de herramientas para analizar el funcionamiento y la ejecución, CUDA solo ofrece un conjunto muy básico de herramientas de análisis y depuración. Así pues, muchas de las soluciones a problemas que han ido surgiendo han tenido que ser resueltas por prueba y error, un enfoque bastante poco óptimo.

El enfoque a lo largo del trabajo de implementar todas las opciones posibles en el tiempo dado tiene la desventaja de no contar con una implementación asentada y limpia, si no una en constante cambio y consecuentemente más complicada de analizar. Aún así, este enfoque ha hecho posible abarcar todas las bases de un motor de producción mínimo, y aunque en ciertas secciones carece de trasfondo teórico, da una visión global de todos los componentes necesarios.

\section{Trabajo Futuro}
	
Eleven Renderer cuenta con las características visuales mínimas de un motor de producción, así pues deja la puerta abierta a un futuro desarrollo comercial o con fines educativos. No obstante, el siguiente desafío si se quiere seguir produciendo, es adaptar la implementación a las suites de software 3D a través de plugins que enlacen las escenas con el motor. 

Queda también una larga lista de mejoras, entre ellas:

\begin{itemize}
	
	\item Mayor variedad de shaders (además del shader de Disney).
	\item Port oneAPI.
	\item Incluir refracción, puesto que por el momento los materiales transparentes son incompatibles con el motor.
	\item Implementar eliminadores de ruido como Open Image Denoise o NVIDIA OptiX™ AI-Accelerated Denoiser.
	\item Añadir un motor de postprocesado más potente con filtros como Bloom o correcciones de color.
	\item Añadir más formatos de luces, como luces de área.
	
\end{itemize}

Queda pendiente realizar a fondo un análisis exhaustivo de eficiencia. En el trabajo se han dado pinceladas en este aspecto, por ejemplo la comparación entre búsquedas recursivas o iterativas en GPU, pero hay un centenar de posibles optimizaciones y de prácticas recomendadas para exprimir la eficiencia de la programación en paralelo.

\chapter*{Conclusions}

Graphic acceleration focused software development can be an easy task at first. Getting the most juice out of this kind of architecture is where the real hard work is, the strengths and limitations must be taken into account.

Referring to Eleven Render development of a photorealistic renderer, there have been many challenges. At first, graphic rendering has a huge theoretical and mathematical background which should be known to be able to add any optimization. This work has just talked about the implementation and evaluation of this software, without doing any emphasis in the theoretical bases. As was mentioned before, Physically based rendering: From theory to implementation \cite{pharr2016physically} is an excellent resource for this, it's even called "The PBR Bible" by the community. 

Secondly, Eleven Renderer coding has been more tedious than expected. While some improvements like the Acceleration Structures building can be seen as discouraging and hard to implement because of their complexity, they were easier than expected in comparison to simpler and subtle improvements. This is because of the lack of options in GPU debugging. While CPU development has a big bag of tools to analyze the execution of CPU software, CUDA just offers basic debugging tools. That being said, many of the problems have been addressed by try and failure, which is a suboptimal approach.

The approach for developing the Renderer incrementally, adding all the possible features, has the disadvantage of not providing a clear and settled implementation, but a changing one which is harder to analyze. Even so, this approach has made it possible to cover all the required bases for a small production renderer, and although in certain sections it lacks a theoretical background, it provides a global view from all the components.

\section{Future work}

Eleven Renderer has the minimal visual features for a production renderer, having said that, it leaves the door open to a further commercial development or with educational purposes. Nevertheless, the next challenge is to adapt the implementation to already working 3D software suites, through plugins which handles the 3D scenes and adapts it to Eleven Renderer.

There's also a big list of future improvements:

\begin{itemize}
	
	\item More shaders (apart from Disney).
	\item oneAPI port.
	\item Adding refraction, as currently the materials are not compatible with transmission.
	\item Implement any noise reduction software, like Open Image Denoise or NVIDIA OptiX™ AI-Accelerated Denoiser.
	\item Add more post processing effects like bloom.
	\item Add more lights like area lights.
	
\end{itemize}

Remains pending an exhaustive analysis of performance. In this work, some benchmarking has been done like the GPU search comparaison (iterative vs recursive), but there's hundreds of possible optimizations and recommended practices to squish every drop of performance from the parallel architectures.

\chapter{Anexo}
	
\section{Manual}
	
Tras compilar el repositorio o haber descargado un ejecutable, es preciso incluir en la línea de comandos tres parámetros. El primero es el directorio donde se encuentra la escena, el segundo es el número de muestras deseado y el tercero el archivo de salida en formato .bmp de la imagen resultante.
	
\subsection{Formato de escenas}
\label{sceneformat}

Debido a la falta de consenso en cuanto a formatos en la industria, se ha utilizado un formato de escenas intentando respetar los estándares más comunes. Así pues una escena se define como un directorio.

Dentro este directorio debe incluir en su interior 3 carpetas

\begin{itemize}
	\item Objects: utilizada para las geometrías en formato .obj. Es necesario que estas geometrías hayan sido trianguladas previamente puesto que el parser desarrollado está limitado a triángulos.
	
	\item Textures: utilizada para las texturas. Actualmente solo se permiten texturas para los atributos: Albedo, Emission, Roughness, Metallic, Normal. Las texturas tienen que estar en formato bmp de 24 bits. El formato de nombre es el siguiente: nombredelmaterial\_tipodemapa.bmp. Por ejemplo: material1\_albedo.bmp, material1\_metallic.bmp. No es necesario que se definan todas las texturas, si no existe alguna se ignorará y se utilizará el valor de color definido en el archivo scene.json. En caso de no existir tampoco ese valor, se utilizará el valor por defecto. Las texturas de albedo y emisión deberán encontrarse en espacio de color sRGB mientras que el resto de texturas deben estar en un espacio lineal. Esto no es respetado por muchos motores de renderizado y es dependiente de la implementación.
	
	\item HDRI: utilizada para los mapas de entorno. Dentro albergará los archivos en formato .hdr de los mapas de entorno.
\end{itemize}

Además será obligatorio incluir un archivo llamado scene.json. Este archivo ha de incluir la información de la escena necesaria. Se muestra un ejemplo como plantilla:

\begin{minipage}[c]{0.95\textwidth}
\begin{lstlisting}
	
{	
'camera' : {'xRes' : 1280, 'yRes' : 720, 'position' : {x : 0, y : 1, z : 2}, 'focalLength' : 0.05, 'focusDistance' : 1, 'aperture' : 2.8},
'materials' : [{'name' : 'mat1'}, {'name' : 'mat2', 'albedo' : {'r' : 1, 'g' : 0, 'b' : 0}, 'roughness' : 0.2}],
'objects' : [{'name' : 'obj1', 'material' : 'mat1'}, {'name' : 'obj2', 'material' : 'mat2'}],
'hdri' : {'name' : 'hdri', 'xOffset' : 0.5},
'pointLights' : [{'position' : {'x' : 0, 'y': 0, 'z': 0}, 'radiance' {'r' : 1, 'g' : 0, 'b' : 0}}]
}
	
\end{lstlisting}
\end{minipage}

El ejemplo mostrado deberá contener dos objetos dentro de la carpeta Objects: obj1.obj y obj2.obj. El material mat1 utilizará las texturas que empiecen por mat1\_...bmp mientras que el mat2 utilizará los valores rgb(1,0,0) para albedo y el valor 0.2 para roughness. Finalmente, se añade una luz puntual en la posición (0,0,0).

Para más ejemplos consultar la carpeta \emph{scenes} del repositorio.

\subsection{Ejemplo de renderizado de escena \emph{ClockCC0}}

Con el fin de demostrar el funcionamiento con un ejemplo real, se ha preparado una escena para la cual, todos los recursos tienen licencia CC0. Esta escena se encuentra en el repositorio y puede ser recreada si se siguen los pasos mostrados en este punto.

Los modelos y texturas utilizados para esta escena son los siguientes:

\begin{itemize}
	\item Modelo de planta: \url{https://polyhaven.com/a/potted_plant_04}
	\item Modelo de mesa: \url{https://polyhaven.com/a/vintage_wooden_drawer_01}
	\item Modelo de reloj: \url{https://polyhaven.com/a/alarm_clock_01}
	\item HDRI de interior: \url{https://polyhaven.com/a/photo_studio_london_hall}
\end{itemize}

Para la ejecución del programa, en el entorno utilizado, ha sido necesario utilizar los siguientes argumentos:
	\code{eleven.exe ''C:$\backslash\backslash$Users$\backslash\backslash$Kike$\backslash\backslash$Desktop$\backslash\backslash$Uni$\backslash\backslash$TFG$\backslash\backslash$Scenes$\backslash\backslash$ClockCC0'' output.bmp 5000}

Tras comenzar el renderizado aparecen dos ventanas, la primera es la consola de salida que se muestra en la \autoref{fig:debugwindow}. Esta ventana contiene datos relevantes en cuanto a la ejecución, como: El progreso de la construcción del BVH, cantidad de memoria utilizada por las geometrías/texturas, eficiencia en kPaths/s, cuanta memoria consume la escena en la GPU y cuanta hay libre, cuántas muestras se han tomado y el tiempo total de ejecución.

Por otro lado, está la ventana de previsualización \autoref{fig:samplewindow}. Esta ventana muestra el progreso visual del render y en la esquina superior izquierda, el número de muestras tomadas por cada píxel hasta el momento.

Finalmente la imagen será guardada como "render.bmp" tras ejecutar las 5000 muestras. En este caso la salida es la \autoref{fig:finalimagewindow}. Este resultado se ha obtenido tras 20 minutos y 39 segundos en una NVIDIA RTX 3090. Con un número mucho menor de muestras es posible seguir obteniendo buenos resultados.

\begin{figure}[H]
    \centering
	\includegraphics[width=0.9\textwidth]{execdebug}
	\caption{Ventana de depuración}
	\label{fig:debugwindow}
\end{figure}

\begin{figure}[H]
    \centering
	\includegraphics[width=0.9\textwidth]{execsamples}
	\caption{Ventana de previsualización}
	\label{fig:samplewindow}
\end{figure}

\begin{figure}[H]
    \centering
	\includegraphics[width=0.9\textwidth]{execfinalimage}
	\caption{Imagen final}
	\label{fig:finalimagewindow}
\end{figure}
	

\bibliographystyle{ieeetr}
\bibliography{refs}

\chapter*{Glosario}
\markboth{GLOSARIO}{GLOSARIO}

\begin{itemize}
	
	\item \textbf{Renderer} Software dedicado a la visualización de una escena digital.
	\item \textbf{PBR} Physically Based Rendering. Renderizado simulando las propiedades físicas de los materiales, el modelo de sombreado está basado en el modelo de microfacetas.
	\item \textbf{BRDF} Bidirectional reflectance distribution function. Función que determina la pérdida de energía de un camino a partir de las direcciones de entrada y salida.
	\item \textbf{PDF} Probability distribution function. Función que indica la probabilidad de una distribución concreta.
	\item \textbf{CDF} Cumulative distribution function. Función que contiene la probabilidad acumulada a lo largo de su dominio.
	\item \textbf{Trazado de rayos} Familia de algoritmos basados en la simulación de algún tipo de rayo.
	
\end{itemize}

 
\end{document}