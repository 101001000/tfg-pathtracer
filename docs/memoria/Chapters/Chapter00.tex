\chapter*{Resumen}
\markboth{RESUMEN}{RESUMEN}

En este trabajo se explica la implementación de Eleven Renderer, un motor de renderizado gráfico programado en C++ y CUDA de código abierto. Cuenta con todas las características necesarias para visualizar una escena 3D con relativa eficiencia y que hace uso de la arquitectura paralela que ofrecen los aceleradores gráficos. Además se acompaña con una evaluación y justificación de los detalles de la implementación además de ofrecer recursos en caso de querer ampliar información. Así pues, se espera que este trabajo sirva como breve guía y motivación para futuros programadores gráficos.

Eleven Renderer permite importar escenas desde la gran mayoría de software de edición 3D, aunque requiere de un proceso manual para adaptar el formato de la escena. Cuenta con una versión en oneAPI que puede ejecutarse en distintas plataformas sin requerir el uso de tarjetas gráficas de NVIDIA.


\chapter*{Abstract}
\markboth{ABSTRACT}{ABSTRACT}


\chapter*{Agradecimientos}
\markboth{AGRADECIMIENTOS}{AGRADECIMIENTOS}


\newpage

\includegraphics[
    width=1.45\textwidth,
    height=2\textwidth,
    align=t,
    smash=br,
    vshift=4cm,    
    hshift=-4.1cm
]{portada}

\scalebox{5}{\color{white}{Eleven Renderer}}

