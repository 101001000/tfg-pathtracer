\chapter{Fundamentos del algoritmo de Path Tracing}
	

En este capítulo se procede a dar un esquema básico de los fundamentos de este algoritmo. El objetivo será así describir un motor de renderizado previo a cualquier optimización, que cuenta con las funcionalidades básicas para producir una imagen de una escena tridimensional simple.

	\section{Aproximación de la ecuación de renderizado}
\[
{\displaystyle L_{\text{o}}(\mathbf {x} ,\omega _{\text{o}})=L_{\text{e}}(\mathbf {x} ,\omega _{\text{o}})\ +\int _{\Omega }f_{\text{r}}(\mathbf {x} ,\omega _{\text{i}},\omega _{\text{o}})L_{\text{i}}(\mathbf {x} ,\omega _{\text{i}})(\omega _{\text{i}}\cdot \mathbf {n} )\operatorname {d} \omega _{\text{i}}}
\]

La ecuación de renderizado \cite{therenderingequation} aparece por primera vez en 1986 junto al algoritmo de Path Tracing, siendo este algoritmo una propuesta para resolverla. Es el pilar de la visualización 3d fotorrealista ya que simula de una manera suficientemente precisa la interacción de la luz.

La interpretación de esta ecuación es la siguiente: Para un punto $\mathbf {x}$ del espacio y un ángulo $\omega _{\text{o}}$ desde el cual se observa a dicho punto, cuál es la cantidad de energía lumínica que el observador recibe $L_{\text{o}}$.

El primer término $L_{\text{e}}(\mathbf {x} ,\omega _{\text{o}})$ indica la luz que dicho punto $\mathbf {x}$ emite, así pues se podrán modelar materiales que emitan luz propia y no dependan de energía externa.

El segundo término calcula toda la luz entrante y reflejada a través del ángulo $\omega _{\text{o}}$ por dicho punto $\mathbf {x}$, es por ello que integra todos los ángulos del hemisferio superior. Este segundo término se compone de tres coeficientes:

El primer coeficiente $f_{\text{r}}(\mathbf {x} ,\omega _{\text{i}},\omega _{\text{o}})$ es la función BRDF, la cual es dependiente del material e indica cuánta energía se refleja en dicho punto para las direcciones de entrada $\omega _{\text{i}}$ y salida $\omega _{\text{o}}$.

El segundo coeficiente $L_{\text{i}}(\mathbf {x} ,\omega _{\text{i}})$ hace referencia a toda la energía lumínica entrante de todas las direcciones posibles.

El tercer coeficiente $(\omega _{\text{i}}\cdot \mathbf {n})$ es el producto de la ley del coseno de Lambert \cite{photometria}, un escalar que atenúa los ángulos menos pronunciados con la normal de la superficie.


	\section{Esquema simplificado del trazado de rayos}
	
Habiendo definido la ecuación de renderizado, el siguiente paso es explicar el algoritmo de Path Tracing en su forma más elemental, ya que posteriormente se irá mejorando paso por paso este algoritmo básico.

En su esencia, este algoritmo consiste en trazar rayos o "caminos" desde una cámara virtual a una escena tridimensional, simulando así un modelo simplificado de fotones y sus interacciones con la escena, las cuales conllevarán una pérdida de energía de estos.


También se utiliza el término Ray Tracing, puesto que este término ha terminado usándose para distintos tipos de simulación de rayos de luz, aunque originalmente Ray Tracing se utilizaba para referirse al algoritmo original propuesto en , el cual difiere el algoritmo analizado en este trabajo. 


El primer paso es preparar la escena a renderizar \code{Scene}. Una escena básica se compone de una cámara, geometrías y materiales.

Las cámaras \code{Camera} consisten en una simulación aproximada de una cámara física real, así pues sus atributos serán: tamaño del sensor (en milímetros) \code{sensorWidth} y \code{sensorHeight}, distancia focal (en milímetros) \code{focalLength} y resolución (en píxeles) \code{xRes} e \code{yRes}. 

Las geometrías \code{MeshObject} por otra parte consisten en un conjunto de triángulos \code{Tri}, los cuales a su vez consisten en 3 puntos tridimensionales \code{Vector3 vertices[3]}.

Los materiales \code{Material} definen la manera en la que los fotones interactúan con ellos. En su forma más primitiva consisten en un color base, el cual absorberá ciertas longitudes de onda en mayor o menor medida. Para simplificar las computaciones, no es necesario calcular estas interacciones con todo el espectro electromagnético visible, basta con usar los tres colores primarios aditivos: rojo, verde y azul.


Así pues, habiendo definido estos elementos en la escena, se procederá a transferirlos a la GPU, donde se realizará el resto de computaciones más demandantes. 


	\subsection{Trazado de rayo desde la cámara}

Para simular el trazado del rayo desde la cámara hasta la escena, se simula de manera simplificada como funcionaría una cámara estenopéica. Se calcula la posición del sensor por la cual se trazará el rayo a partir de las coordenadas \code{x} e \code{y}. 

Puesto que no se está teniendo en cuenta la rotación de la cámara, la coordenada z del sensor se puede simplificar con la distancia de la cámara hasta el sensor.

\begin{lstlisting}
	
__device__ void calculateCameraRay(int x, int y, Camera& camera, Ray& ray, float r1, float r2, float r3, float r4, float r5) {

    // Relative coordinates for the point where the first ray will be launched
    float dx = camera.position.x + ((float)x) / ((float)camera.xRes) * camera.sensorWidth;
    float dy = camera.position.y + ((float)y) / ((float)camera.yRes) * camera.sensorHeight;

    // Absolute coordinates for the point where the first ray will be launched
    float odx = (-camera.sensorWidth / 2.0) + dx;
    float ody = (-camera.sensorHeight / 2.0) + dy;

    // Random part of the sampling offset so we get antialasing
    float rx = (1.0 / (float)camera.xRes) * (r1 - 0.5) * camera.sensorWidth;
    float ry = (1.0 / (float)camera.yRes) * (r2 - 0.5) * camera.sensorHeight;

    // Sensor point, the point where intersects the ray with the sensor
    float SPx = odx + rx;
    float SPy = ody + ry;
    float SPz = camera.position.z + camera.focalLength;

    // The initial ray is created from the camera position to the sensor point. No rotation is taken into account.
    ray = Ray(camera.position, Vector3(SPx, SPy, SPz) - camera.position);
}

\end{lstlisting}

	\subsection{Intersección triángulo - rayo}
	\label{subsec:triintersection}
	
El cálculo de la intersección de un rayo con un triángulo es una de las operaciones más fundamentales de este algoritmo. Esta operación toma como parámetros un triángulo \code{Tri} y un rayo \code{Ray} y ofrece como resultado si dicho triángulo y rayo intersectan en el espacio y además un objeto \code{Hit} el cual cuenta con información adicional de la intersección.

\code{bool hit(Ray\& ray, Hit\& hit)}

Para la implementación se ha hecho uso del algoritmo Fast Minimum Storage Ray/Triangle Intersection\cite{triintersection}. Este paper también ofrece una implementación en C, la cual ha sido adaptada al proyecto.

La información adicional que devuelve esta operación es la siguiente:

	\code{int hit.objectID}: ID del objecto al que pertenece dicho triángulo.
	
	\code{Vector3 hit.position}: Posición en el espacio del punto de intersección entre el rayo y el triángulo.
	
	\code{Vector3 hit.normal}: Normal de la superficie, calculada a partir del producto vectorial de dos aristas del triángulo.
	
	\code{bool hit.valid}: Validez de una intersección, por defecto falso. Verdadero en caso de haber intersectado correctamente.







	










