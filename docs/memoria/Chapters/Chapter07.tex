\chapter{Anexo}
	
\section{Manual}
	
Tras compilar el repositorio o haber descargado un ejecutable, es preciso incluir en la línea de comandos tres parámetros. El primero es el directorio donde se encuentra la escena, el segundo es el número de muestras deseado y el tercero el archivo de salida en formato .bmp de la imagen resultante.
	
\subsection{Formato de escenas}
\label{sceneformat}

Debido a la falta de consenso en cuanto a formatos en la industria, se ha utilizado un formato de escenas intentando respetar los estándares más comunes. Así pues una escena se define como un directorio.

Dentro este directorio debe incluir en su interior 3 carpetas

\begin{itemize}
	\item Objects: utilizada para las geometrías en formato .obj. Es necesario que estas geometrías hayan sido trianguladas previamente puesto que el parser desarrollado está limitado a triángulos.
	
	\item Textures: utilizada para las texturas. Actualmente solo se permiten texturas para los atributos: Albedo, Emission, Roughness, Metallic, Normal. Las texturas tienen que estar en formato bmp de 24 bits. El formato de nombre es el siguiente: nombredelmaterial\_tipodemapa.bmp. Por ejemplo: material1\_albedo.bmp, material1\_metallic.bmp. No es necesario que se definan todas las texturas, si no existe alguna se ignorará y se utilizará el valor de color definido en el archivo scene.json. En caso de no existir tampoco ese valor, se utilizará el valor por defecto. Las texturas de albedo y emisión deberán encontrarse en espacio de color sRGB mientras que el resto de texturas deben estar en un espacio lineal. Esto no es respetado por muchos motores de renderizado y es dependiente de la implementación.
	
	\item HDRI: utilizada para los mapas de entorno. Dentro albergará los archivos en formato .hdr de los mapas de entorno.
\end{itemize}

Además será obligatorio incluir un archivo llamado scene.json. Este archivo ha de incluir la información de la escena necesaria. Se muestra un ejemplo como plantilla:

\begin{lstlisting}
	
{	
'camera' : {'xRes' : 1280, 'yRes' : 720, 'position' : {x : 0, y : 1, z : 2}, 'focalLength' : 0.05, 'focusDistance' : 1, 'aperture' : 2.8},
'materials' : [{'name' : 'mat1'}, {'name' : 'mat2', 'albedo' : {'r' : 1, 'g' : 0, 'b' : 0}, 'roughness' : 0.2}],
'objects' : [{'name' : 'obj1', 'material' : 'mat1'}, {'name' : 'obj2', 'material' : 'mat2'}],
'hdri' : {'name' : 'hdri', 'xOffset' : 0.5},
}
	
\end{lstlisting}

El ejemplo mostrado deberá contener dos objetos dentro de la carpeta Objects: obj1.obj y obj2.obj. El material mat1 utilizará las texturas que empiecen por mat1\_...bmp mientras que el mat2 utilizará los valores rgb(1,0,0) para albedo y el valor 0.2 para roughness. Para más ejemplos consultar la carpeta Scenes del repositorio.

\section{Galería}
	